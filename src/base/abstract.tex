% This file contains the abstract of the thesis

Multiple Wireless Connectivity is a technology that allows a mobile device to
connect and transmit data on multiple Wi-Fi connections by using multiple virtual
interfaces mapped on the same physical Wi-Fi card. Nowadays, it is more and
more common for a mobile device to use in parallel both a Wi-Fi Direct
connection (aka P2P) and a regular Wi-Fi 802.11 connection. Also, research
technologies like Multi-WiFi shows that using multiple 802.11 Wi-Fi connections
in parallel improves the user-experience in terms of throughput and delay.

This master project is split in two directions. The first one is represented by the identification
and analysis of the existing solutions for Multiple Wireless Connectivity on mobile
devices with a focus on the power consumption perspective. The second direction
is represented by the design and implementation of an algorithm that reduces the energy consumption
on mobile devices when multiple Wi-Fi connections are used in parallel. In this
thesis I show that the existing solutions for Multiple Wireless Connectivity
lack any optimizations for energy consumption. Starting from this observation,
I show that my algorithm can drop the energy consumption by up to 50 percent
while keeping the user-experience at an acceptable level.





 

